\vspace{-1mm}
\section{Conclusion}
\vspace{-1mm}

Inspired by the learning behavior of humans when they try to acquire a new skill, we propose {\topic} that trains an agent from books, instead of numerous interaction data with the real environment. We also implement a practical algorithm of understanding, rehearsing, and introspecting modules to realize such a learning paradigm. The result of deploying our method in a football game environment demonstrates a huge improvement in the winning rate over the baseline methods. This success proves the feasibility of utilizing knowledge stored in various written texts for decision-making agents, which was neglected by the community for a long time. 

We hope that this promising result will initiate more research on {\topic} and we leave more discussion of the limitation and several interesting open problems of this study in Appendix~\ref{app:limi}.

 % further generalize the learning paradigm to materials of other modalities and deploy it in more scenarios.

% \section{Discussion and Open Problems}

% We hope this promising result will initiate more research on {\topic}, including better knowledge extraction and representation, combining it with traditional trial-and-error methods, or introducing other planning modules into the framework. In the following, we would also like to list several open problems for consideration:

% In light of our proposed framework, \algo, for extracting knowledge from textual sources, it is pertinent to consider whether it represents the most effective pipeline currently available. For example, the manner in which knowledge is represented is of paramount importance, prompting us to question whether coding is indeed the most efficient medium for this purpose.

% Acknowledging the synergistic relationship between reading and practical experience in human learning, we are prompted to explore how machines might integrate interaction data into their learning processes. This hybridization of learning strategies may significantly enhance the agent’s decision-making capabilities.

% Furthermore, given the prevalence of multi-modal tutorials in the real world, incorporating multimedia materials such as images and videos into the framework will be a promising next-step study. This expansion will enable a more comprehensive understanding of the subject matter and enrich the learning experience for the agent.

% We are also compelled to investigate other potential applications of the framework beyond the scope of the current experiment. Domains such as cooking and chess, which require both theoretical knowledge and practical skill, could benefit significantly from the integration of the learning paradigm. This exploration opens up exciting avenues for future research and application of the method.




